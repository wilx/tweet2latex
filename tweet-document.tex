\documentclass[a4paper,twocolumn]{article}
\usepackage[paper=A4,DIV=calc,twocolumn]{typearea}
\usepackage{ifxetex,ifluatex}
\usepackage{fontspec}
\usepackage{graphicx,grffile}
\usepackage{etoolbox}
\usepackage[unicode=true]{hyperref}
\urlstyle{same}  % don't use monospace font for urls
\hypersetup{breaklinks=true, hidelinks}
\usepackage{bookmark}
\usepackage{float}
\usepackage{wrapfig}
\usepackage{ragged2e}
\usepackage{tcolorbox}
\usepackage{xcolor}
\usepackage{parskip}

\usepackage{polyglossia}
\setdefaultlanguage{english}
\setotherlanguages{czech}
\def\czechfont{}

\defaultfontfeatures{Ligatures=TeX,Scale=MatchLowercase}
\setsansfont{TeX Gyre Heros}

\ifxetex
\usepackage{xeCJK}
\setCJKfamilyfont{jarm}{Noto Sans CJK JP}

%% The following breaks Japanese text.
%\usepackage{ucharclasses}
%\setTransitionTo{Emoticons}{\begingroup \fontspec{Symbola} }
%\setTransitionFrom{Emoticons}{\endgroup }
\else
\usepackage{luatexja-fontspec}
\usepackage{luatexja-preset}
\setmainjfont{Noto Sans CJK JP}[Scale=MatchUppercase]
\fi

\usepackage{microtype}

\setlength{\emergencystretch}{3em}  % prevent overfull lines

\makeatletter
\def\maxwidth{\ifdim\Gin@nat@width>\linewidth\linewidth\else\Gin@nat@width\fi}
\def\maxheight{\ifdim\Gin@nat@height>\textheight\textheight\else\Gin@nat@height\fi}
\makeatother
\setkeys{Gin}{width=\maxwidth,height=\maxheight,keepaspectratio}

\tcbuselibrary{breakable, skins}

\newenvironment{tweet}{%
  \newcommand{\tweetUserImage}[3]{%
    \begingroup%
      \includegraphics[keepaspectratio,height=1em]{##2}%
      \quad
    \endgroup
  }%
  \newcommand{\tweetUserName}[3]{\href{https://twitter.com/intent/user?user_id=##1}{##2}\quad
    \href{https://twitter.com/intent/user?user_id=##1}{{\small \color{gray}@##3}}\\}%
  \newcommand{\tweetHashtag}[2]{\href{https://twitter.com/hashtag/##1}{##2}}%
  \newcommand{\tweetUserMention}[2]{\href{https://twitter.com/intent/user?user_id=##1}{##2}}%
  \newcommand{\tweetUrl}[4]{\href{##2}{##3}}%
  \newcommand{\tweetPhoto}[4]{\\\includegraphics[keepaspectratio]{##3}\\}%
  \newcommand{\tweetItself}[4]{\flushright
    \href{https://twitter.com/statuses/##1}{{\small \color{gray}##3 ##4}}}%
  \newcommand{\tweetPlace}[2]{\flushright {\small \color{gray}##1, ##2}}%
  \newfontfamily\emojifont{Symbola}[Scale=MatchUppercase]%
  \begin{tcolorbox}[size=small,enhanced,breakable,autoparskip,halign=flush left]%
    \sffamily%
}{\end{tcolorbox}}

% This allows line breaks in URL in more places.
\def\UrlBreaks{\do\/\do-\do.\do=\do_\do?\do\&\do\%\do\a\do\b\do\c\do\d\do\e\do\f\do\g\do\h\do\i\do\j\do\k\do\l\do\m\do\n\do\o\do\p\do\q\do\r\do\s\do\t\do\u\do\v\do\w\do\x\do\y\do\z\do\A\do\B\do\C\do\D\do\E\do\F\do\G\do\H\do\I\do\J\do\K\do\L\do\M\do\N\do\O\do\P\do\Q\do\R\do\S\do\T\do\U\do\V\do\W\do\X\do\Y\do\Z\do\0\do\1\do\2\do\3\do\4\do\5\do\6\do\7\do\8\do\9}


\begin{document}

\title{Testing Tweets}
\author{Author's Name}

\maketitle

\begin{abstract}
  This is a test of (semi-)automated tweets formatting. This started as an
  attempt to answer \TeX .SE question
  \href{http://tex.stackexchange.com/q/323562/28495}{``Quoting tweets inside
    a \LaTeX{} document?''} See
  \texttt{\href{https://github.com/wilx/tweet2latex}{tweet2latex}} GitHub
  repository for source code and more details.
\end{abstract}

\section{The tweets}\label{the-tweets}%
Here are some tweets:
\begin{tweet}\tweetUserImage{https://pbs.twimg.com/profile\_images/629633358641922048/gZLP7aJg\_normal.jpg}{gZLP7aJg-normal.jpg}{1359055698}\tweetUserName{1359055698}{Václav Haisman}{vzeman79}\tweetHashtag{justatest}{\#justatest} url: \tweetUrl{https://t.co/maTGizVFyp}{http://tex.stackexchange.com/questions/323562/quoting-tweets-inside-a-latex-document}{tex.stackexchange.com/questions/3235…}{https://t.co/maTGizVFyp}\tweetItself{762697549979484161}{Mon Aug 08 17:10:40 +0000 2016}\end{tweet}


\begin{tweet}\tweetUserImage{https://pbs.twimg.com/profile\_images/968256400655753217/fWAI6OGF\_bigger.jpg}{fWAI6OGF-bigger.jpg}{9083222}\tweetUserName{9083222}{Sam Riegel}{samriegel}\tweetUserVerified{}\tweetUserEnd{}\tweetHashtag{ShirtCeption}{\#ShirtCeption} continues.... tonight! \tweetHashtag{CriticalRole}{\#CriticalRole} \tweetPhoto{https://twitter.com/samriegel/status/758761004255711232/photo/1}{https://pbs.twimg.com/media/CoepG82VMAAMPi_.jpg}{CoepG82VMAAMPi-.jpg}{https://t.co/j8Q4mUlrWy}\tweetRetweets{306}\tweetFavorites{2032}\tweetItself{758761004255711232}{Thu Jul 28 20:28:15 +0000 2016}{July 28, 2016}{10:28:15 PM GMT+2}\end{tweet}

\begin{tweet}\tweetUserImage{https://pbs.twimg.com/profile\_images/683232086958993408/rnyugqzL\_normal.jpg}{rnyugqzL-normal.jpg}{701158958}\tweetUserName{701158958}{MedicNow}It could be worse. You could be the lifeguard at the \tweetHashtag{Rio}{\#Rio} swimming pool.... \tweetHashtag{MondayMotivation}{\#MondayMotivation} \tweetPhoto{http://twitter.com/MedicNow/status/762602474293321728/photo/1}{https://pbs.twimg.com/media/CpVOzW7WEAAhMte.jpg}{CpVOzW7WEAAhMte.jpg}{https://t.co/AfoOoV9qQw}\tweetItself{762602474293321728}{Mon Aug 08 10:52:52 +0000 2016}\end{tweet}


\begin{tweet}\tweetUserImage{https://pbs.twimg.com/profile\_images/432081479/DOI\_LOGO\_bigger.jpg}{DOI-LOGO-bigger.jpg}{76348185}\tweetUserName{76348185}{US Dept of Interior}{Interior}Sunsets don't get much better than this one over \tweetUserMention{44991932}{@GrandTetonNPS}. \tweetHashtag{nature}{\#nature} \tweetHashtag{sunset}{\#sunset} \tweetPhoto{http://twitter.com/Interior/status/463440424141459456/photo/1}{https://pbs.twimg.com/media/Bm54nBCCYAACwBi.jpg}{Bm54nBCCYAACwBi.jpg}{http://t.co/YuKy2rcjyU}\tweetItself{463440424141459456}{Mon May 05 22:09:42 +0000 2014}{May 6, 2014}{12:09:42 AM GMT+2}\end{tweet}


\begin{tweet}\tweetUserImage{https://pbs.twimg.com/profile\_images/714370334951096320/Pd4EWLl\_\_normal.jpg}{Pd4EWLl--normal.jpg}{508192094}\tweetUserName{508192094}{Laura Bailey}On our way to \tweetHashtag{GenCon}{\#GenCon}!! Listening to my Vex playlist to get ready for the craziness. :) \tweetHashtag{CriticalRole}{\#CriticalRole} 
\tweetUrl{https://t.co/UjdKx0uAzB}{http://geekandsundry.com/critical-role-vexs-soundtrack-a-rhapsody-in-arrows/}{geekandsundry.com/critical-role-…}{https://t.co/UjdKx0uAzB}\tweetItself{761310539280769024}{Thu Aug 04 21:19:11 +0000 2016}\end{tweet}


\section{More tweets}\label{more-tweets}%
And here are some more tweets:
\begin{tweet}\tweetUserImage{https://pbs.twimg.com/profile\_images/652629100406947840/tQJVTFB9\_normal.jpg}{tQJVTFB9-normal.jpg}{1857463891}\tweetUserName{1857463891}{Sean Carney}As summer rolls on, thoughts turn to refreshing beverages. Czech Saaz Hops a Hit With Asian Brewers   \tweetUrl{https://t.co/iZUWfebOY6}{http://on.wsj.com/154pCj7}{on.wsj.com/154pCj7}{https://t.co/iZUWfebOY6} via \tweetUserMention{3108351}{@WSJ}\tweetItself{762738381143023620}{Mon Aug 08 19:52:55 +0000 2016}\end{tweet}


\begin{tweet}\textczech{\tweetUserImage{https://pbs.twimg.com/profile\_images/378800000050569517/281e318f9889350d70c894ca6bc729a9\_bigger.jpeg}{281e318f9889350d70c894ca6bc729a9-bigger.jpeg}{575208808}\tweetUserName{575208808}{Vaclav Stetka}{VStetka}Reportéři bez hranic kritizují koncentraci moci A.Babiše / Local oligarch conflicts of interest dominate Czech media \tweetUrl{https://t.co/lz3pjtMVvf}{https://rsf.org/en/news/local-oligarch-conflicts-interest-dominate-czech-media}{rsf.org/en/news/local-…}{https://t.co/lz3pjtMVvf}}\tweetItself{757989653412777984}{Tue Jul 26 17:23:10 +0000 2016}{July 26, 2016}{7:23:10 PM GMT+2}\end{tweet}


\begin{tweet}\tweetUserImage{https://pbs.twimg.com/profile\_images/762642310928687104/sMutHhUO\_normal.jpg}{sMutHhUO-normal.jpg}{326832066}\tweetUserName{326832066}{Alastair Reynolds}{AquilaRift}I've mentioned it already, so not letting any cats out of bags, but working on sequel to The Prefect. Set two years on.\tweetItself{762629946099568641}{Mon Aug 08 12:42:02 +0000 2016}{August 8, 2016}{2:42:02 AM GMT+2}\end{tweet}


\tweetCaption{761779435653652480}{「ネット保守連合」事務局 たかすぎ}{nihonjintamasii}{Sat Aug 06 04:22:25 +0000 2016}{August 6, 2016}{6:22:25 AM GMT+2}{}{}{}\begin{tweet}\tweetUserImage{https://pbs.twimg.com/profile\_images/1124955009/PHOT000000000005BCAD\_500\_0\_bigger.jpg}{PHOT000000000005BCAD-500-0-bigger.jpg}{122999362}\tweetUserName{122999362}{「ネット保守連合」事務局 たかすぎ}{nihonjintamasii}\tweetUserEnd{}【靖国神社参拝】\hfill\break
\null{}日本国を守るために亡くなられた、英霊の方々を慰霊するのに\hfill\break
\null{}反日国家の韓国に気をつかう必要など一切無い\hfill\break
\null{}\hfill\break
\null{}岸田外務大臣の\hfill\break
\null{}慰安婦問題の「日韓合意」や「靖国神社参拝拒否」は\hfill\break
\null{}大臣も国会議員も失格\hfill\break
\null{}人間として信用できない \tweetPhoto{https://twitter.com/nihonjintamasii/status/761779435653652480/photo/1}{https://pbs.twimg.com/media/CpJhqQBUsAAk669.jpg}{CpJhqQBUsAAk669.jpg}{https://t.co/tX1XUPARqe}\tweetRetweets{475}\tweetFavorites{283}\tweetItself{761779435653652480}{Sat Aug 06 04:22:25 +0000 2016}{August 6, 2016}{6:22:25 AM GMT+2}\end{tweet}

\section{Last two tweets}\label{last-two-tweets}%
Still not satisfied?
\begin{tweet}\tweetUserImage{https://pbs.twimg.com/profile\_images/183983583/weinstein200-1\_bigger.jpg}{weinstein200-1-bigger.jpg}{33152005}\tweetUserName{33152005}{Eric Weinstein}{EricRWeinstein}It would appear that ISIS may be about to ramp up terror ops out of their frustration w/ the PC left's denial of any terror-scripture link.\tweetItself{760888006022574081}{Wed Aug 03 17:20:11 +0000 2016}{August 3, 2016}{7:20:11 PM GMT+2}\end{tweet}

\begin{tweet}\tweetUserImage{https://pbs.twimg.com/profile\_images/827205023155183616/Uk3aswVz\_bigger.jpg}{Uk3aswVz-bigger.jpg}{2780874870}\tweetUserName{2780874870}{Christopher Perkins}{ChrisPerkinsDnD}\tweetUserEnd{}Confused. What's your gaming question? {\emojifont 😚} \tweetUrl{https://t.co/DpmargYSy6}{https://twitter.com/bezno1/status/763817959009689600}{twitter.com/bezno1/status/…}{https://t.co/DpmargYSy6}\tweetRetweets{1}\tweetFavorites{76}\tweetItself{763841870480879617}{Thu Aug 11 20:57:48 +0000 2016}{August 11, 2016}{10:57:48 PM GMT+2}\tweetPlace{Renton, WA}{United States}{https://www.google.com/maps/search/?query=Renton\%2C+WA\%2C+United+States\&api=1}\end{tweet}

\section{I am spent!}\label{i-am-spent}%
\begin{tweet}\tweetUserImage{https://pbs.twimg.com/profile\_images/1124955009/PHOT000000000005BCAD\_500\_0\_bigger.jpg}{PHOT000000000005BCAD-500-0-bigger.jpg}{122999362}\tweetUserName{122999362}{「ネット保守連合」事務局 たかすぎ}{nihonjintamasii}【靖国神社参拝】\hfill\break
\null{}日本国を守るために亡くなられた、英霊の方々を慰霊するのに\hfill\break
\null{}反日国家の韓国に気をつかう必要など一切無い\hfill\break
\null{}\hfill\break
\null{}岸田外務大臣の\hfill\break
\null{}慰安婦問題の「日韓合意」や「靖国神社参拝拒否」は\hfill\break
\null{}大臣も国会議員も失格\hfill\break
\null{}人間として信用できない \tweetPhoto{https://twitter.com/nihonjintamasii/status/761779435653652480/photo/1}{https://pbs.twimg.com/media/CpJhqQBUsAAk669.jpg}{CpJhqQBUsAAk669.jpg}{https://t.co/tX1XUPARqe}\tweetItself{761779435653652480}{Sat Aug 06 04:22:25 +0000 2016}{August 6, 2016}{6:22:25 AM GMT+2}\end{tweet}

\begin{tweet}\tweetUserImage{https://pbs.twimg.com/profile\_images/856890499894038529/8YeBhFD4\_bigger.jpg}{8YeBhFD4-bigger.jpg}{17537467}\tweetUserName{17537467}{Tarek Fatah}{TarekFatah}India's Urdu papers urge country's Muslims to wage \tweetHashtag{Jihad}{\#Jihad} against non-Muslims \char`\~{} \tweetUserMention{21002587}{@TufailElif}. \tweetUrl{https://t.co/elpMfZi8yV}{http://www.openthemagazine.com/article/society/call-for-jihad\#all}{openthemagazine.com/article/societ…}{https://t.co/elpMfZi8yV} \tweetPhoto{https://twitter.com/TarekFatah/status/762518022372352004/photo/1}{https://pbs.twimg.com/media/CpUCFe5WIAEuiNb.jpg}{CpUCFe5WIAEuiNb.jpg}{https://t.co/1HFD3khMuk}\tweetItself{762518022372352004}{Mon Aug 08 05:17:18 +0000 2016}{August 8, 2016}{7:17:18 AM GMT+2}\tweetPlace{Toronto, Ontario}{Canada}\end{tweet}


\begin{tweet}\tweetUserImage{https://pbs.twimg.com/profile\_images/694970119831818240/X6SS1zC3\_bigger.jpg}{X6SS1zC3-bigger.jpg}{17537467}\tweetUserName{17537467}{(((Tarek Fatah)))}{TarekFatah}India's Urdu papers urge country's Muslims to wage \tweetHashtag{Jihad}{\#Jihad} against non-Muslims \char`\~{} \tweetUserMention{21002587}{@TufailElif}. \tweetUrl{https://t.co/elpMfZi8yV}{http://www.openthemagazine.com/article/society/call-for-jihad\#all}{openthemagazine.com/article/societ…}{https://t.co/elpMfZi8yV} \tweetPhoto{http://twitter.com/TarekFatah/status/762518022372352004/photo/1}{https://pbs.twimg.com/media/CpUCFe5WIAEuiNb.jpg}{CpUCFe5WIAEuiNb.jpg}{https://t.co/1HFD3khMuk}\tweetItself{762518022372352004}{Mon Aug 08 05:17:18 +0000 2016}{August 8, 2016}{7:17:18 AM GMT+2}\tweetPlace{Toronto, Ontario}{Canada}\end{tweet}


\begin{tweet}\tweetUserImage{https://pbs.twimg.com/profile\_images/974345186124865536/BOFmvdKD\_bigger.jpg}{BOFmvdKD-bigger.jpg}{34048973}\tweetUserName{34048973}{eternal void-posting}{blargler}\tweetUserEnd{}\tweetInReplyToTweet{866548991613214721}{1733467884}{shoe0nhead}\tweetUserMention{1733467884}{@shoe0nhead}  \tweetPhoto{https://twitter.com/blargler/status/866553062361415681/photo/1}{https://pbs.twimg.com/media/DAadaFgUwAEvxnS.jpg}{DAadaFgUwAEvxnS.jpg}{https://t.co/THfuX1KbOr}\tweetRetweets{92}\tweetFavorites{1059}\tweetItself{866553062361415681}{Mon May 22 07:15:24 +0000 2017}{May 22, 2017}{9:15:24 AM GMT+2}\end{tweet}

\begin{tweet}\textczech{\tweetUserImage{https://pbs.twimg.com/profile\_images/661920080092127232/8omkVgyg\_bigger.png}{8omkVgyg-bigger.png}{338672002}\tweetUserName{338672002}{ČT sport}{sportCT}Třetí medaile z tenisových kurtů v Riu! \tweetUserMention{1465525608}{@stepec78} a \tweetUserMention{89692547}{@lucik2105} mají \tweetHashtag{Bronze}{\#Bronze}, gratulujeme! {\emojifont 👏👍👏}\hfill\break
\null{}\tweetHashtag{czechteam}{\#czechteam}🇨🇿 \tweetHashtag{Tennis}{\#Tennis} \tweetPhoto{https://twitter.com/sportCT/status/764859108579274752/photo/1}{https://pbs.twimg.com/media/Cp1SuiwXEAAVvGD.jpg}{Cp1SuiwXEAAVvGD.jpg}{https://t.co/uPjhqXsrIS}}\tweetItself{764859108579274752}{Sun Aug 14 16:19:56 +0000 2016}{August 14, 2016}{6:19:56 PM GMT+2}\end{tweet}


This tweet is hilarious:
\begin{tweet}\tweetUserImage{https://pbs.twimg.com/profile\_images/628503398069841920/bTeea\_pG\_bigger.jpg}{bTeea-pG-bigger.jpg}{34048973}\tweetUserName{34048973}{Intermattercourse}{blargler}\tweetUserMention{1733467884}{@shoe0nhead}  \tweetPhoto{https://twitter.com/blargler/status/866553062361415681/photo/1}{https://pbs.twimg.com/media/DAadaFgUwAEvxnS.jpg}{DAadaFgUwAEvxnS.jpg}{https://t.co/THfuX1KbOr}\tweetItself{866553062361415681}{Mon May 22 07:15:24 +0000 2017}{May 22, 2017}{9:15:24 AM GMT+2}\end{tweet}


\section{Not stopping here!}\label{not-stopping-here}%
\begin{tweet}\textczech{\tweetUserImage{https://pbs.twimg.com/profile\_images/661920080092127232/8omkVgyg\_bigger.png}{8omkVgyg-bigger.png}{338672002}\tweetUserName{338672002}{ČT sport}{sportCT}Třetí medaile z tenisových kurtů v Riu! \tweetUserMention{1465525608}{@stepec78} a \tweetUserMention{89692547}{@lucik2105} mají \tweetHashtag{Bronze}{\#Bronze}, gratulujeme! {\emojifont 👏👍👏}\hfill\break
\null{}\tweetHashtag{czechteam}{\#czechteam}🇨🇿 \tweetHashtag{Tennis}{\#Tennis} \tweetPhoto{http://twitter.com/sportCT/status/764859108579274752/photo/1}{https://pbs.twimg.com/media/Cp1SuiwXEAAVvGD.jpg}{Cp1SuiwXEAAVvGD.jpg}{https://t.co/uPjhqXsrIS}}\tweetItself{764859108579274752}{Sun Aug 14 16:19:56 +0000 2016}{August 14, 2016}{6:19:56 PM GMT+2}\end{tweet}


\end{document}
